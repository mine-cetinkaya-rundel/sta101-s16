% -*- TeX-engine: xetex; eval: (auto-fill-mode 0); eval: (visual-line-mode 1); -*-
% Compile with XeLaTeX

%%%%%%%%%%%%%%%%%%%%%%%
% Option 1: Slides: (comment for handouts)   %
%%%%%%%%%%%%%%%%%%%%%%%

\documentclass[slidestop,compress,mathserif,12pt,t,professionalfonts,xcolor=table]{beamer}

% solution stuff
\newcommand{\solnMult}[1]{
\only<1>{#1}
\only<2->{\red{\textbf{#1}}}
}
\newcommand{\soln}[1]{\textit{#1}}

%%%%%%%%%%%%%%%%%%%%%%%%%%%%%%%
% Option 2: Handouts, without solutions (post before class)    %
%%%%%%%%%%%%%%%%%%%%%%%%%%%%%%%

% \documentclass[11pt,containsverbatim,handout,xcolor=xelatex,dvipsnames,table]{beamer}
%
% % handout layout
% \usepackage{pgfpages}
% \pgfpagesuselayout{4 on 1}[letterpaper,landscape,border shrink=5mm]
%
%% % solution stuff
% \newcommand{\solnMult}[1]{#1}
% \newcommand{\soln}[1]{}

%%%%%%%%%%%%%%%%%%%%%%%%%%%%%%%%%%%%
% Option 3: Handouts, with solutions (may post after class if need be)    %
%%%%%%%%%%%%%%%%%%%%%%%%%%%%%%%%%%%%

% \documentclass[11pt,containsverbatim,handout,xcolor=xelatex,dvipsnames,table]{beamer}

% % handout layout
% \usepackage{pgfpages}
% \pgfpagesuselayout{4 on 1}[letterpaper,landscape,border shrink=5mm]

% % solution stuff
% \newcommand{\solnMult}[1]{\red{\textbf{#1}}}
% \newcommand{\soln}[1]{\textit{#1}}

%%%%%%%%%%
% Load style file, defaults  %
%%%%%%%%%%

%%%%%%%%%%%%%%%%
% Themes
%%%%%%%%%%%%%%%%

% See http://deic.uab.es/~iblanes/beamer_gallery/ for mor options

% Style theme
\usetheme{Pittsburgh}

% Color theme
\usecolortheme{seahorse}

% Helvetica Neue Light for most text
\usepackage{fontspec}
\setsansfont{Helvetica Neue Light}

%%%%%%%%%%%%%%%%
% Packages
%%%%%%%%%%%%%%%%

\usepackage{geometry}
\usepackage{graphicx}
\usepackage{amssymb}
\usepackage{epstopdf}
\usepackage{amsmath}  	% this permits text in eqnarray among other benefits
\usepackage{url}		% produces hyperlinks
\usepackage[english]{babel}
\usepackage{colortbl}	% allows for color usage in tables
\usepackage{multirow}	% allows for rows that span multiple rows in tables
\usepackage{color}		% this package has a variety of color options
\usepackage{pgf}
\usepackage{calc}
\usepackage{ulem}
\usepackage{multicol}
\usepackage{textcomp}
\usepackage{listings}
\usepackage{changepage}
\usepackage{tikz}
\usetikzlibrary{trees}		% for probability trees
\usepackage{fancyvrb}	% for colored code chunks
\usepackage{nameref}

%%%%%%%%%%%%%%%%
% Remove navigation symbols
%%%%%%%%%%%%%%%%

\beamertemplatenavigationsymbolsempty
\hypersetup{pdfpagemode=UseNone} % don't show bookmarks on initial view

%%%%%%%%%%%%%%%%
% User defined colors
%%%%%%%%%%%%%%%%

% Pantone 2016 Spring colors
% https://atelierbram.github.io/c-tiles16/colorscheming/pantone-spring-2016-colortable.html
% update each semester or year

\xdefinecolor{custom_blue}{rgb}{0.01, 0.31, 0.52} % Snorkel Blue
\xdefinecolor{custom_darkBlue}{rgb}{0.20, 0.20, 0.39} % Reflecting Pond  
\xdefinecolor{custom_orange}{rgb}{0.96, 0.57, 0.42} % Cadmium Orange
\xdefinecolor{custom_green}{rgb}{0, 0.47, 0.52} % Biscay Bay
\xdefinecolor{custom_red}{rgb}{0.58, 0.32, 0.32} % Marsala

\xdefinecolor{custom_lightGray}{rgb}{0.78, 0.80, 0.80} % Glacier Gray
\xdefinecolor{custom_darkGray}{rgb}{0.35, 0.39, 0.43} % Stormy Weather

%%%%%%%%%%%%%%%%
% Template colors
%%%%%%%%%%%%%%%%

\setbeamercolor*{palette primary}{fg=white,bg= custom_blue}
\setbeamercolor*{palette secondary}{fg=black,bg= custom_blue!80!black}
\setbeamercolor*{palette tertiary}{fg=white,bg= custom_blue!80!black!80}
\setbeamercolor*{palette quaternary}{fg=white,bg= custom_blue}

\setbeamercolor{structure}{fg= custom_blue}
\setbeamercolor{frametitle}{bg= custom_blue!90}
\setbeamertemplate{blocks}[shadow=false]
\setbeamersize{text margin left=2em,text margin right=2em}

%%%%%%%%%%%%%%%%
% Styling fonts, bullets, etc.
%%%%%%%%%%%%%%%%

% title slide
\setbeamerfont{title}{size=\large,series=\bfseries}
\setbeamerfont{subtitle}{size=\large,series=\mdseries}
%\setbeamerfont{institute}{size=\large,series=\mdseries}

% color of alerted text
\setbeamercolor{alerted text}{fg=custom_orange}

% styling of itemize bullets
\setbeamercolor{item}{fg=custom_blue}
\setbeamertemplate{itemize item}{{{\small$\blacktriangleright$}}}
\setbeamercolor{subitem}{fg=custom_blue}
\setbeamertemplate{itemize subitem}{{\textendash}}
\setbeamerfont{itemize/enumerate subbody}{size=\footnotesize}
\setbeamerfont{itemize/enumerate subitem}{size=\footnotesize}

% styling of enumerate bullets
\setbeamertemplate{enumerate item}{\insertenumlabel.}
\setbeamerfont{enumerate item}{family={\fontspec{Helvetica Neue}}}
\setbeamerfont{enumerate subitem}{family={\fontspec{Helvetica Neue}}}
\setbeamerfont{enumerate subsubitem}{family={\fontspec{Helvetica Neue}}}

% make frame titles small to make room in the slide
\setbeamerfont{frametitle}{size=\small} 

% set Helvetica Neue font for frame and section titles
\setbeamerfont{frametitle}{family={\fontspec{Helvetica Neue}}}
\setbeamerfont{sectiontitle}{family={\fontspec{Helvetica Neue}}}
\setbeamerfont{section in toc}{family={\fontspec{Helvetica Neue}}}
\setbeamerfont{subsection in toc}{family={\fontspec{Helvetica Neue}}, size=\small}
\setbeamerfont{footline}{family={\fontspec{Helvetica Neue}}}
\setbeamerfont{subsection in toc}{family={\fontspec{Helvetica Neue}}}
\setbeamerfont{block title}{family={\fontspec{Helvetica Neue}}}

%%%%%%%%%%%%%%%%
% New fonts accessed by fontspec package
%%%%%%%%%%%%%%%%

% Monaco font for code
\newfontfamily{\monaco}{Monaco}

%%%%%%%%%%%%%%%%
% Color text commands
%%%%%%%%%%%%%%%%

%orange
\newcommand{\orange}[1]{\textit{\textcolor{custom_orange}{#1}}}

% yellow
\newcommand{\yellow}[1]{\textit{\textcolor{yellow}{#1}}}

% blue
\newcommand{\blue}[1]{\textit{\textcolor{blue}{#1}}}

% green
\newcommand{\green}[1]{\textit{\textcolor{custom_green}{#1}}}

% red
\newcommand{\red}[1]{\textit{\textcolor{custom_red}{#1}}}

% dark gray
\newcommand{\darkgray}[1]{\textit{\textcolor{custom_darkGray}{#1}}}

% light gray
\newcommand{\lightgray}[1]{\textit{\textcolor{custom_lightGray}{#1}}}

% pink
\newcommand{\pink}[1]{\textit{\textcolor{pink}{#1}}}


%%%%%%%%%%%%%%%%
% Custom commands
%%%%%%%%%%%%%%%%

% empty box for probability tree frame
\newcommand{\emptybox}[2]{
	\fbox{ \begin{minipage}{#1} \hfill\vspace{#2} \end{minipage} }
}

% cancel
\newcommand{\cancel}[1]{%
    \tikz[baseline=(tocancel.base)]{
        \node[inner sep=0pt,outer sep=0pt] (tocancel) {#1};
        \draw[red, line width=0.5mm] (tocancel.south west) -- (tocancel.north east);
    }%
}

% degree
\newcommand{\degree}{\ensuremath{^\circ}}

% cite
\newcommand{\ct}[1]{
\vfill
{\tiny #1}}

% Note
\newcommand{\Note}[1]{
\rule{2.5cm}{0.25pt} \\ \textit{\footnotesize{\textcolor{custom_red}{Note:} \textcolor{custom_darkGray}{#1}}}}

% Remember
\newcommand{\Remember}[1]{\textit{\scriptsize{\textcolor{custom_red}{Remember:} #1}}}

% links: webURL, webLink
\newcommand{\webURL}[1]{\urlstyle{same}{\textit{\textcolor{custom_blue}{\url{#1}}}}}
\newcommand{\webLink}[2]{\href{#1}{\textcolor{custom_blue}{{#2}}}}

% mail
\newcommand{\mail}[1]{\href{mailto:#1}{\textit{\textcolor{custom_blue}{#1}}}}

% highlighting: hl, hlGr, mathhl
\newcommand{\hl}[1]{\textit{\textcolor{custom_blue}{#1}}}
\newcommand{\hlGr}[1]{\textit{\textcolor{custom_green}{#1}}}
\newcommand{\mathhl}[1]{\textcolor{custom_blue}{\ensuremath{#1}}}

% example
\newcommand{\ex}[1]{\textcolor{blue}{{{\small (#1)}}}}

% twocol: two columns
\newenvironment{twocol}[4]{
\begin{columns}[c]
\column{#1\textwidth}
#3
\column{#2\textwidth}
#4
\end{columns}
}

% threecol: three columns
\newenvironment{threecol}[6]{
\begin{columns}[c]
\column{#1\textwidth}
#4
\column{#2\textwidth}
#5
\column{#3\textwidth}
#6
\end{columns}
}

% slot (for probability calculations)
\newenvironment{slot}[2]{
\begin{array}{c} 
\underline{#1} \\ 
#2
\end{array}
}

% pr: left and right parentheses
\newcommand{\pr}[1]{
\left( #1 \right)
}

%%%%%%%%%%%%%%%%
% Custom blocks
%%%%%%%%%%%%%%%%

% activity: less commonly used
\newcommand{\activity}[2]{
\setbeamertemplate{itemize item}{{{\small\textcolor{custom_orange}{$\blacktriangleright$}}}}
\setbeamercolor{block title}{fg=white, bg=custom_orange}
\setbeamerfont{block title}{size=\small}
\setbeamercolor{block body}{fg=black, bg=custom_orange!20!white!80}
\setbeamerfont{block body}{size=\small}
\begin{block}{Activity: #1}
\setlength\abovedisplayskip{0pt}
#2
\end{block}
}

% app: application exercise
\newcommand{\app}[2]{
\setbeamercolor{block title}{fg=white,bg=custom_green}
\setbeamercolor{block body}{fg=black,bg=custom_green!20!white!80}
\begin{block}{{\small Application exercise: #1}}
#2
\end{block}
}

% disc: discussion question
\newcommand{\disc}[1]{
\vspace*{-2ex}
\setbeamercolor{block body}{bg=custom_blue!25!white!80, fg=custom_blue!55!black!95}
\begin{block}{\vspace*{-3ex}}
#1
\end{block}
\vspace*{-1ex}
}

% clicker: clicker question
\newcommand{\clicker}[1]{
\setbeamercolor{block title}{bg=custom_blue!80!white!50,fg=custom_blue!30!black!90}
\setbeamercolor{block body}{bg=custom_blue!20!white!80,fg=custom_blue!30!black!90}
\begin{block}{\vspace*{-0.2ex}{\footnotesize Clicker question}\vspace*{-0.2ex}}
#1
\end{block}
}

% formula
\newcommand{\formula}[2]{
\setbeamercolor{block title}{bg=custom_blue!40!white!60,fg=custom_blue!55!black!95}
\begin{block}{{\small#1}}
#2
\end{block}
}

% code
\newcommand{\Rcode}[1]{
{\monaco {\footnotesize \textcolor{custom_darkBlue}{#1}}}
}

% output
\newcommand{\Rout}[1]{
{\monaco {\footnotesize \textcolor{custom_darkGray}{#1}}}
}

%%%%%%%%%%%%%%%%
% Change margin
%%%%%%%%%%%%%%%%

\newenvironment{changemargin}[2]{%
\begin{list}{}{%
\setlength{\topsep}{0pt}%
\setlength{\leftmargin}{#1}%
\setlength{\rightmargin}{#2}%
\setlength{\listparindent}{\parindent}%
\setlength{\itemindent}{\parindent}%
\setlength{\parsep}{\parskip}%
}%
\item}{\end{list}}

%%%%%%%%%%%%%%%%
% Footnote
%%%%%%%%%%%%%%%%

\long\def\symbolfootnote[#1]#2{\begingroup%
\def\thefootnote{\fnsymbol{footnote}}\footnote[#1]{#2}\endgroup}

%%%%%%%%%%%%%%%%
% Graphics
%%%%%%%%%%%%%%%%

\DeclareGraphicsRule{.tif}{png}{.png}{`convert #1 `dirname #1`/`basename #1 .tif`.png}

%%%%%%%%%%%%%%%%
% Slide number
%%%%%%%%%%%%%%%%

\setbeamertemplate{footline}{%
    \raisebox{5pt}{\makebox[\paperwidth]{\hfill\makebox[20pt]{\color{gray}
          \scriptsize\insertframenumber}}}\hspace*{5pt}}

          
%%%%%%%%%%%%%%%%
% Remove page numbers
%%%%%%%%%%%%%%%%

\newcommand{\removepagenumbers}{% 
  \setbeamertemplate{footline}{}
}

%%%%%%%%%%%%%%%%
% TOC slides
%%%%%%%%%%%%%%%%

\setbeamertemplate{section in toc}{\inserttocsectionnumber.~\inserttocsection}
\setbeamertemplate{subsection in toc}{$\qquad$\inserttocsubsectionnumber.~\inserttocsubsection \\}

\AtBeginSection[] 
{ 
  \addtocounter{framenumber}{-1} 
  % 
  {\removepagenumbers 
  {\small
    \begin{frame}<beamer> 
    \frametitle{Outline} 
    \tableofcontents[currentsection] 
  \end{frame} 
  } 
  }
} 

\AtBeginSubsection[] 
{ 
  \addtocounter{framenumber}{-1} 
  % 
  {\removepagenumbers 
  {\small
    \begin{frame}<beamer> 
    \frametitle{Outline} 
    \tableofcontents[currentsection,currentsubsection] 
  \end{frame} 
  } 
  }
}
% Course Name
\newcommand{\CourseName}{Sta 101 - Spring 2016}
\newcommand{\InstituteName}{Duke University, Department of Statistical Science}

% Personal Info
\newcommand{\FirstName}{Mine}
\newcommand{\LastName}{\c{C}etinkaya-Rundel}

% Electronic Info
\newcommand{\PersonalSite}{http://stat.duke.edu/~mc301}
\newcommand{\CourseSite}{http://bit.ly/sta101_s16}
\newcommand{\Email}{mine@stat.duke.edu}

% Exam Dates
\newcommand{\ExamADate}{Feb 24, Wed}
\newcommand{\ExamBDate}{Mar 30, Wed}
\newcommand{\FinalDate}{May 5, Thu - 7-10pm}
% ALT ALT
% % Course Name
\newcommand{\CourseName}{Sta 101 - Spring 2016}
\newcommand{\InstituteName}{Duke University, Department of Statistical Science}

% Personal Info
\newcommand{\FirstName}{Anthea}
\newcommand{\LastName}{Monod}

% Electronic Info
\newcommand{\PersonalSite}{https://stat.duke.edu/people/anthea-monod.html}
\newcommand{\CourseSite}{https://stat.duke.edu/courses/Spring16/sta101.002}
\newcommand{\Email}{anthea@stat.duke.edu}

% Exam Dates
\newcommand{\ExamADate}{Feb 25, Thu}
\newcommand{\ExamBDate}{Mar 31, Thu}
\newcommand{\FinalDate}{???}

%%%%%%%%%%%
% Cover slide info    %
%%%%%%%%%%%

\title{Unit 7: Multiple linear regression}
\subtitle{3. Transformations \& case study}
\author{\CourseName}
\date{}
\institute{\InstituteName}


%%%%%%%%%%%%%%%%%%%%%%%%%
% Begin document and set Helvetica Neue font   %
%%%%%%%%%%%%%%%%%%%%%%%%%

\begin{document}
\fontspec[Ligatures=TeX]{Helvetica Neue Light}

%%%%%%%%%%%%%%%%%%%%%%%%%%%%%%%%%%%

% Title Page

\begin{frame}[plain]

\titlepage

\vfill

{\scriptsize \webLink{\PersonalSite}{Dr. \LastName{}} \hfill Slides posted at  \webURL{\CourseSite}}

\addtocounter{framenumber}{-1} 

\end{frame}

%%%%%%%%%%%%%%%%%%%%%%%%%%%%%%%%%%%%

\section{Housekeeping}

%%%%%%%%%%%%%%%%%%%%%%%%%%%%%%%%%%%%

\begin{frame}
\frametitle{Announcements}

\begin{itemize}

\item Poster session on Thursday -- see email for details, come to The Edge Workshop Room during
your regular lab time

\item I won't hold OH on Thursday (during the poster sessions)

\end{itemize}

\end{frame}

%%%%%%%%%%%%%%%%%%%%%%%%%%%%%%%%%%%%

\section{Transformations}

%%%%%%%%%%%%%%%%%%%%%%%%%%%%%%%%%%%%

\begin{frame}[fragile]
\frametitle{Truck prices}

\disc{The scatterplot below shows the relationship between year and price of a random sample of 43 pickup trucks. Describe the relationship between these two variables.}

\begin{center}
\includegraphics[width=0.7\textwidth]{figures/pickup/pu_price_year_scat_allyrs}
\end{center}

\ct{From: \webURL{http://faculty.chicagobooth.edu/robert.gramacy/teaching.html}}


\end{frame}

%%%%%%%%%%%%%%%%%%%%%%%%%%%%%%%%%%%

\begin{frame}[fragile]
\frametitle{Remove unusual observations}

Let's remove trucks older than 20 years, and only focus on trucks made in 1992 or later.

\pause

\disc{Now what can you say about the relationship?}

\begin{center}
\includegraphics[width=0.7\textwidth]{figures/pickup/pu_price_year_scat_noline}
\end{center}

\end{frame}

%%%%%%%%%%%%%%%%%%%%%%%%%%%%%%%%%%%

\begin{frame}
\frametitle{Truck prices - linear model?}

\hl{Model:}  $\widehat{price} = b_0 + b_1~year$

$\:$ \\

\includegraphics[width=0.5\textwidth]{figures/pickup/pu_price_year_scat} 
\includegraphics[width=0.5\textwidth]{figures/pickup/pu_price_year_res}

\pause

$\:$ \\

The linear model doesn't appear to be a good fit since the residuals have non-constant variance.


\end{frame}

%%%%%%%%%%%%%%%%%%%%%%%%%%%%%%%%%%%

\begin{frame}
\frametitle{Truck prices - log transform of the response variable}

\hl{Model:} $\widehat{log(price)} = b_0 + b_1~year$

$\:$ \\

\includegraphics[width=0.5\textwidth]{figures/pickup/pu_price_year_scat_log} 
\includegraphics[width=0.5\textwidth]{figures/pickup/pu_price_year_res_log}

\pause

$\:$ \\

We applied a log transformation to the response variable. The relationship now seems linear, and the residuals 
no longer have non-constant variance.

\end{frame}

%%%%%%%%%%%%%%%%%%%%%%%%%%%%%%%%%%%

\begin{frame}
\frametitle{Interpreting models with log transformation}

\begin{center}
\begin{tabular}{rrrrr}
  \hline
 & Estimate & Std. Error & t value & Pr($>$$|$t$|$) \\ 
  \hline
(Intercept) & -265.073 & 25.042 & -10.585 & 0.000 \\ 
  year & 0.137 & 0.013 & 10.937 & 0.000 \\ 
   \hline
\end{tabular}
\end{center}

\pause
\[ \hl{\text{Model: }} \widehat{log(price)} = -265.073 + 0.137~year \]

\pause

\begin{itemize}

\item For each additional year the car is newer (for each year decrease in car's age) we would expect the log 
price of the car to increase on average by 0.137 log dollars.

\pause

\item which is not very useful...

\end{itemize}

\end{frame}

%%%%%%%%%%%%%%%%%%%%%%%%%%%%%%%%%%%

\begin{frame}
\frametitle{Working with logs}

\begin{itemize}

\item Subtraction and logs: $log(a) - log(b) = log(\frac{a}{b})$

\pause

\item Natural logarithm: $e^{log(x)} = x$

\pause

\item We can these identities to ``undo" the log transformation 

\end{itemize}


\end{frame}


%%%%%%%%%%%%%%%%%%%%%%%%%%%%%%%%%%%

\begin{frame}
\frametitle{Interpreting models with log transformation (cont.)}

The slope coefficient for the log transformed model is 0.137, meaning the \underline{log} price difference between cars that are one year apart is predicted to be 0.14 log dollars.

\begin{eqnarray*}
\text{log(price at year $x + 1$)} - \text{log(price at year $x$)} &=& 0.137 \\
\pause
log \pr{ \frac{\text{price at year $x + 1$}}{\text{price at year $x$}} } &=& 0.137 \\
\pause
e^{log \pr{ \frac{\text{price at year $x + 1$}}{\text{price at year $x$}} }} &=& e^{0.137} \\
\pause
\frac{\text{price at year $x + 1$}}{\text{price at year $x$}} &=& 1.15
\end{eqnarray*}

\pause

For each additional year the car is newer (for each year decrease in car's age) we would expect the price of the car to increase on average \red{by a factor of 1.15}.

\end{frame}

%%%%%%%%%%%%%%%%%%%%%%%%%%%%%%%%%%%

\begin{frame}
\frametitle{Recap: dealing with non-constant variance}

\begin{itemize}

\item Non-constant variance is one of the most common model violations, however it is usually fixable by transforming the response ($y$) variable

\pause

\item The most common variance stabilizing transform is the log transformation: $log(y)$, especially useful when the response variable is (extremely) right skewed.

\pause

\item When using a log transformation on the response variable the interpretation of the slope changes: \pause
\begin{itemize}
\item For each unit increase in $x$, y is expected on average to decrease/increase by a factor of $e^{b_1}$.
\end{itemize}

\pause

\item Another useful transformation is the square root: $\sqrt{y}$, especially useful when the response variable is counts.

\pause

\item These transformations may also be useful when the relationship is non-linear, but in those cases a polynomial regression may also be needed.

\end{itemize}

\end{frame}

%%%%%%%%%%%%%%%%%%%%%%%%%%%%%%%%%%%

\section{Case study}

%%%%%%%%%%%%%%%%%%%%%%%%%%%%%%%%%%%

\begin{frame}
\frametitle{Data from the ACS}

{\scriptsize
\begin{enumerate}
\item \texttt{income}: Yearly income (wages and salaries)
\item \texttt{employment}: Employment status, not in labor force, unemployed, or employed
\item \texttt{hrs\_work}: Weekly hours worked
\item \texttt{race}: Race, White, Black, Asian, or other
\item \texttt{age}: Age
\item \texttt{gender}: gender, male or female
\item \texttt{citizens}: Whether respondent is a US citizen or not
\item \texttt{time\_to\_work}: Travel time to work
\item \texttt{lang}: Language spoken at home, English or other
\item \texttt{married}: Whether respondent is married or not
\item \texttt{edu}: Education level, hs or lower, college, or grad
\item \texttt{disability}: Whether respondent is disabled or not
\item \texttt{birth\_qrtr}: Quarter in which respondent is born, jan thru mar, apr thru jun, jul thru sep, or oct thru dec 
\end{enumerate}
}

\end{frame}

%%%%%%%%%%%%%%%%%%%%%%%%%%%%%%%%%%%

\begin{frame}[fragile]
\frametitle{Load and subset data}

{\footnotesize
\begin{Verbatim}[frame=single, formatcom=\color{blue}]
acs_emp <- acs %>%
  filter(employment == "employed", income > 0)
\end{Verbatim}
}

\end{frame}

%%%%%%%%%%%%%%%%%%%%%%%%%%%%%%%%%%%

\begin{frame}[fragile]
\frametitle{Aside: categorical (factor) variables in R}

{\footnotesize
\begin{Verbatim}[frame=single, formatcom=\color{blue}]
acs_emp %>%
  select(employment) %>%
  table()
\end{Verbatim}

\pause

\begin{Verbatim}[frame=single, formatcom=\color{gray}]
not in labor force         unemployed           employed 
                 0                  0                787
\end{Verbatim}

\pause

\begin{Verbatim}[frame=single, formatcom=\color{blue}]
acs_emp <- droplevels(acs_emp) # overwrite acs_emp

acs_emp %>%
  select(employment) %>%
  table()
\end{Verbatim}

\begin{Verbatim}[frame=single, formatcom=\color{gray}]
employed 
     787 
\end{Verbatim}
}

\end{frame}

%%%%%%%%%%%%%%%%%%%%%%%%%%%%%%%%%%%

\begin{frame}[fragile]
\frametitle{Full model}

Suppose we only want to consider the following explanatory variables: \texttt{hrs\_work}, \texttt{race}, \texttt{age}, \texttt{gender}, \texttt{citizen}.

{\footnotesize
\begin{Verbatim}[frame=single, formatcom=\color{blue}]
m_full = lm(income ~ hrs_work + race + age + gender 
    + citizen, data = acs_emp)
\end{Verbatim}
}

\pause

\begin{center}
{\small
\begin{tabular}{rrrrr}
  \hline
 & Estimate & Std. Error & t value & Pr($>$$|$t$|$) \\ 
  \hline
(Intercept) & -17215.60 & 11399.81 & -1.51 & 0.13 \\ 
  hrs\_work & 1251.31 & 153.14 & 8.17 & 0.00 \\ 
  raceblack & -13202.39 & 6373.05 & -2.07 & 0.04 \\ 
  raceasian & 32699.34 & 8903.66 & 3.67 & 0.00 \\ 
  raceother & -12032.88 & 7556.78 & -1.59 & 0.11 \\ 
  age & 760.99 & 129.71 & 5.87 & 0.00 \\ 
  genderfemale & -17246.91 & 3887.17 & -4.44 & 0.00 \\ 
  citizenyes & -9537.20 & 8360.85 & -1.14 & 0.25 \\ 
   \hline
\end{tabular}
}
\end{center}

\end{frame}

%%%%%%%%%%%%%%%%%%%%%%%%%%%%%%%%%%%

\begin{frame}[fragile]
\frametitle{Diagnostics}

\vspace{-0.5cm}

\disc{What do you think?}

\begin{center}
\includegraphics[width=0.95\textwidth]{figures/acs/diag}
\end{center}


\end{frame}

%%%%%%%%%%%%%%%%%%%%%%%%%%%%%%%%%%%

\begin{frame}[fragile]
\frametitle{Diagnostics -- code}

{\scriptsize
\begin{Verbatim}[frame=single, formatcom=\color{blue}]
# residuals vs. fitted
qplot(data = m_full, y = .resid, x = .fitted, geom = "point") +
  geom_hline(yintercept = 0, linetype = "dashed") +
  xlab("Fitted values") +
  ylab("Residuals") +
  ggtitle("Residuals vs. Fitted values")

# histogram of residuals
qplot(data = m_full, x = .resid, geom = "histogram") +
  xlab("Residuals") +
  ggtitle("Histogram of residuals")

# normal prob plot of residuals
qplot(data = m_full, sample = .resid, stat = "qq") +
  ggtitle("Normal probability plot of residuals")

# order of residuals
qplot(data = m_full, y = .resid) +
  geom_hline(yintercept = 0, linetype = "dashed") +
  ylab("Residuals") +
  xlab("Order of data collection") +
  ggtitle("Residuals vs. Order of data collection")
\end{Verbatim}
}

\end{frame}

%%%%%%%%%%%%%%%%%%%%%%%%%%%%%%%%%%%

\begin{frame}[fragile]
\frametitle{Log transformation}

\vspace{-0.4cm}

{\footnotesize
\begin{Verbatim}[frame=single, formatcom=\color{blue}]
m_full_log = lm(log(income) ~ hrs_work + race + age 
    + gender + citizen, data = acs_emp)
\end{Verbatim}
}

\vspace{-0.3cm}

\begin{center}
\includegraphics[width=0.95\textwidth]{figures/acs/diag_log}
\end{center}


\end{frame}

%%%%%%%%%%%%%%%%%%%%%%%%%%%%%%%%%%%

\begin{frame}
\frametitle{}

\vfill

\app{7.4 Interpreting models with a transformed response}{See course website for more details}

\vfill

\end{frame}

%%%%%%%%%%%%%%%%%%%%%%%%%%%%%%%%%%%

\end{document}
