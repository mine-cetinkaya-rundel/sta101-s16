\documentclass[12pt]{article}
%%%%%%%%%%%%%%%%
% Packages
%%%%%%%%%%%%%%%%

\usepackage[top=1cm,bottom=0.5cm,left=1.5cm,right= 1.5cm]{geometry}
\usepackage[parfill]{parskip}
\usepackage{graphicx, fontspec, xcolor,multicol, enumitem, setspace}
\DeclareGraphicsRule{.tif}{png}{.png}{`convert #1 `dirname #1`/`basename #1 .tif`.png}

%%%%%%%%%%%%%%%%
% No page number
%%%%%%%%%%%%%%%%

\pagestyle{empty}

%%%%%%%%%%%%%%%%
% User defined colors
%%%%%%%%%%%%%%%%

% Pantone 2015 Fall colors
% http://iwork3.us/2015/02/18/pantone-2015-fall-fashion-report/
% update each semester or year

\xdefinecolor{custom_blue}{rgb}{0, 0.32, 0.48} % FROM SPRING 2016 COLOR PREVIEW
\xdefinecolor{custom_darkBlue}{rgb}{0.20, 0.20, 0.39} % Reflecting Pond  
\xdefinecolor{custom_orange}{rgb}{0.96, 0.57, 0.42} % Cadmium Orange
\xdefinecolor{custom_green}{rgb}{0, 0.47, 0.52} % Biscay Bay
\xdefinecolor{custom_red}{rgb}{0.58, 0.32, 0.32} % Marsala

\xdefinecolor{custom_lightGray}{rgb}{0.78, 0.80, 0.80} % Glacier Gray
\xdefinecolor{custom_darkGray}{rgb}{0.35, 0.39, 0.43} % Stormy Weather

%%%%%%%%%%%%%%%%
% Color text commands
%%%%%%%%%%%%%%%%

%orange
\newcommand{\orange}[1]{\textit{\textcolor{custom_orange}{#1}}}

% yellow
\newcommand{\yellow}[1]{\textit{\textcolor{yellow}{#1}}}

% blue
\newcommand{\blue}[1]{\textit{\textcolor{blue}{#1}}}

% green
\newcommand{\green}[1]{\textit{\textcolor{custom_green}{#1}}}

% red
\newcommand{\red}[1]{\textit{\textcolor{custom_red}{#1}}}

%%%%%%%%%%%%%%%%
% Coloring titles, links, etc.
%%%%%%%%%%%%%%%%

\usepackage{titlesec}
\titleformat{\section}
{\color{custom_blue}\normalfont\Large\bfseries}
{\color{custom_blue}\thesection}{1em}{}
\titleformat{\subsection}
{\color{custom_blue}\normalfont}
{\color{custom_blue}\thesubsection}{1em}{}

\newcommand{\ttl}[1]{ \textsc{{\LARGE \textbf{{\color{custom_blue} #1} } }}}

\newcommand{\tl}[1]{ \textsc{{\large \textbf{{\color{custom_blue} #1} } }}}

\usepackage[colorlinks=false,pdfborder={0 0 0},urlcolor= custom_orange,colorlinks=true,linkcolor= custom_orange, citecolor= custom_orange,backref=true]{hyperref}

%%%%%%%%%%%%%%%%
% Instructions box
%%%%%%%%%%%%%%%%

\newcommand{\inst}[1]{
\colorbox{custom_blue!20!white!50}{\parbox{\textwidth}{
	\vskip10pt
	\leftskip10pt \rightskip10pt
	#1
	\vskip10pt
}}
\vskip10pt
}

%%%%%%%%%%%
% App Ex number    %
%%%%%%%%%%%

% DON'T FORGET TO UPDATE

\newcommand{\appno}[1]
{2.1}

%%%%%%%%%%%%%%
% Turn on/off solutions       %
%%%%%%%%%%%%%%

% Off
\newcommand{\soln}[2]{$\:$\\ \vspace{#1}}{}

%%% On
%\newcommand{\soln}[2]{\textit{\textcolor{custom_red}{#2}}}{}

%%%%%%%%%%%%%%%%
% Document
%%%%%%%%%%%%%%%%

\begin{document}
\fontspec[Ligatures=TeX]{Helvetica Neue Light}

Dr. \c{C}etinkaya-Rundel \hfill Sta 101: Data Analysis and Statistical Inference \\
Duke University - Department of Statistical Science \hfill \\

\ttl{Application exercise \appno{}: \\
Voting probabilities of college students}

\inst{$\:$ \\
Team name: \rule{10cm}{0.5pt} \\
$\:$ \\
Lab section: $\qquad$ 8:30 $\qquad$ 10:05 $\qquad$ 11:45 $\qquad$ 1:25 $\qquad$ 3:05 $\qquad$ 4:40 \\
$\:$ \\
Write your responses in the spaces provided below. WRITE LEGIBLY and SHOW ALL WORK! 
Only one submission per team is required. One team will be randomly selected and their 
responses will be discussed and graded. Concise and coherent are best!}

The following table shows the distribution of class year and whether or not students 
voted in the last presidential election for 176 Sta 101 students. \\

\begin{center}
\begin{tabular}{rrrr|r}
  \hline
 & no, eligible but didn't & no, not eligible & yes & total \\ 
  \hline
first-year & 3 & 38 & 3 & 44 \\ 
  sophomore & 10 & 40 & 14 & 64 \\ 
  junior & 7 & 6 & 41 & 54 \\ 
  senior & 4 & 1 & 9 & 14 \\ 
  \hline
  total & 24 & 85 & 67 & 176 \\ 
   \hline
\end{tabular}
\end{center}

Answer the following questions based on these data. Make sure to show all your work. \\

\begin{enumerate}

\item  What is the probability that a randomly chosen student has voted in the last 
presidential election?

\soln{2cm}{P(voted) = 67 / 176 = 0.38}

\item What is the probability that a randomly chosen student is a junior \underline{and} 
has voted in the last presidential election?

\soln{2cm}{P(junior and voted) = 41 / 176 = 0.23}

\item What is the probability that a randomly chosen student  has voted in the last presidential 
election \underline{given that} s/he is a junior?

\soln{2cm}{P(voted~$|$~junior) = 41 / 54 = 0.76}

\item Categorize the three probabilities you calculated above as marginal, conditional, 
or joint.

\soln{2cm}{1. marginal, 2. joint, 3. conditional}

\item What is the probability that a randomly chosen student is a junior \underline{or} 
has voted in the last presidential election?

\soln{2cm}{P(junior or voted) = (67 + 54 - 41) / 176 =  0.45}

\item What percent of students are junior \underline{or} have voted in the last 
presidential election?

\soln{1cm}{Same as above, 45\%.}

\item What is the probability that a randomly chosen student has voted in the last presidential 
election \underline{given that} s/he is a first-year? What about sophomore, and senior?

\soln{2cm}{P(voted~$|$~first-year) = 3 / 44 = 0.07 \\
P(voted~$|$~sophomore) = 14 / 64 = 0.22 \\
P(voted~$|$~senior) = 9 / 14 = 0.64}

\item Do these data suggest an association between class year and whether or not 
students have voted in the last presidential election? Explain your reasoning in one or 
two sentences.

\soln{2cm}{Yes, it does. Likelihood of voting varies by class year.}

\end{enumerate}

\end{document}