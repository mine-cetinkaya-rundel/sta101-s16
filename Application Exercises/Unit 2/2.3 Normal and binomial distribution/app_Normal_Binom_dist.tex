\documentclass[12pt]{article}
%%%%%%%%%%%%%%%%
% Packages
%%%%%%%%%%%%%%%%

\usepackage[top=1cm,bottom=0.5cm,left=1.5cm,right= 1.5cm]{geometry}
\usepackage[parfill]{parskip}
\usepackage{graphicx, fontspec, xcolor,multicol, enumitem, setspace}
\DeclareGraphicsRule{.tif}{png}{.png}{`convert #1 `dirname #1`/`basename #1 .tif`.png}

%%%%%%%%%%%%%%%%
% No page number
%%%%%%%%%%%%%%%%

\pagestyle{empty}

%%%%%%%%%%%%%%%%
% User defined colors
%%%%%%%%%%%%%%%%

% Pantone 2015 Fall colors
% http://iwork3.us/2015/02/18/pantone-2015-fall-fashion-report/
% update each semester or year

\xdefinecolor{custom_blue}{rgb}{0, 0.32, 0.48} % FROM SPRING 2016 COLOR PREVIEW
\xdefinecolor{custom_darkBlue}{rgb}{0.20, 0.20, 0.39} % Reflecting Pond  
\xdefinecolor{custom_orange}{rgb}{0.96, 0.57, 0.42} % Cadmium Orange
\xdefinecolor{custom_green}{rgb}{0, 0.47, 0.52} % Biscay Bay
\xdefinecolor{custom_red}{rgb}{0.58, 0.32, 0.32} % Marsala

\xdefinecolor{custom_lightGray}{rgb}{0.78, 0.80, 0.80} % Glacier Gray
\xdefinecolor{custom_darkGray}{rgb}{0.35, 0.39, 0.43} % Stormy Weather

%%%%%%%%%%%%%%%%
% Color text commands
%%%%%%%%%%%%%%%%

%orange
\newcommand{\orange}[1]{\textit{\textcolor{custom_orange}{#1}}}

% yellow
\newcommand{\yellow}[1]{\textit{\textcolor{yellow}{#1}}}

% blue
\newcommand{\blue}[1]{\textit{\textcolor{blue}{#1}}}

% green
\newcommand{\green}[1]{\textit{\textcolor{custom_green}{#1}}}

% red
\newcommand{\red}[1]{\textit{\textcolor{custom_red}{#1}}}

%%%%%%%%%%%%%%%%
% Coloring titles, links, etc.
%%%%%%%%%%%%%%%%

\usepackage{titlesec}
\titleformat{\section}
{\color{custom_blue}\normalfont\Large\bfseries}
{\color{custom_blue}\thesection}{1em}{}
\titleformat{\subsection}
{\color{custom_blue}\normalfont}
{\color{custom_blue}\thesubsection}{1em}{}

\newcommand{\ttl}[1]{ \textsc{{\LARGE \textbf{{\color{custom_blue} #1} } }}}

\newcommand{\tl}[1]{ \textsc{{\large \textbf{{\color{custom_blue} #1} } }}}

\usepackage[colorlinks=false,pdfborder={0 0 0},urlcolor= custom_orange,colorlinks=true,linkcolor= custom_orange, citecolor= custom_orange,backref=true]{hyperref}

%%%%%%%%%%%%%%%%
% Instructions box
%%%%%%%%%%%%%%%%

\newcommand{\inst}[1]{
\colorbox{custom_blue!20!white!50}{\parbox{\textwidth}{
	\vskip10pt
	\leftskip10pt \rightskip10pt
	#1
	\vskip10pt
}}
\vskip10pt
}

%%%%%%%%%%%
% App Ex number    %
%%%%%%%%%%%

% DON'T FORGET TO UPDATE

\newcommand{\appno}[1]
{2.3}

%%%%%%%%%%%%%%
% Turn on/off solutions       %
%%%%%%%%%%%%%%

% Off
\newcommand{\soln}[2]{$\:$\\ \vspace{#1}}{}

%%% On
%\newcommand{\soln}[2]{\textit{\textcolor{custom_red}{#2}}}{}

%%%%%%%%%%%%%%%%
% Document
%%%%%%%%%%%%%%%%

\begin{document}
\fontspec[Ligatures=TeX]{Helvetica Neue Light}

Dr. \c{C}etinkaya-Rundel \hfill Sta 101: Data Analysis and Statistical Inference \\
Duke University - Department of Statistical Science \hfill \\

\ttl{Application exercise \appno{}: \\
Hourly rates of manufacturing workers}

\inst{$\:$ \\
Team name: \rule{10cm}{0.5pt} \\
$\:$ \\
Lab section: $\qquad$ 8:30 $\qquad$ 10:05 $\qquad$ 11:45 $\qquad$ 1:25 $\qquad$ 3:05 $\qquad$ 4:40 \\
$\:$ \\
Write your responses in the spaces provided below. WRITE LEGIBLY and SHOW ALL WORK! 
Only one submission per team is required. One team will be randomly selected and their 
responses will be discussed and graded. Concise and coherent are best!}

In this activity we'll work with data on average hourly wage for manufacturing workers, in the United States as well as in 
North Carolina. The data come from the The 2012 Statistical Abstract. Assume that the distributions of the manufacturing 
wage rates, nationwide and in North Carolina, can be approximated by a normal distribution. \\

{\footnotesize Source: Source: U.S. Bureau of Labor Statistics, Current Employment Statistics, ``State and Metro Area 
Employment, Hours, and Earnings (SAE), March, 2010, \url{http://www.bls.gov/sae/\#data.htm} and \url{http://www.census.gov/compendia/statab/2012/tables/12s1016.pdf}.} \\

\textbf{Part 1:}
Government data indicates that the average hourly wage for manufacturing workers in the United States is \$18.61, 
with a standard deviation of \$1.35. 
\begin{enumerate}

\item What percent of manufacturing workers make more than \$20/hour?

\soln{3.5cm}{
Given: $X_{US} \sim N(\mu = 18.61, \sigma = 1.35)$
\[ P(X > 20) = P\left( Z > \frac{20 - 18.61}{1.35} \right) = P(Z > 1.02) = 0.154 \rightarrow 15.4\% \]
}

\item What percent of manufacturing workers make between \$18 - \$20/hour? \\

\soln{3.5cm}{
\[ P(X_{US} > 18) = P\left( Z > \frac{18 - 18.61}{1.35} \right) = P(Z > -0.45) = 0.674 \]
\[ P(18 < X_{US} < 20) = 0.674 - 0.154 = 0.52 \rightarrow 52\% \]
}

\end{enumerate}

\pagebreak

\textbf{Part 2:}
Government data also indicates that the average hourly wage for manufacturing workers in North Carolina is \$15.85. 

\begin{enumerate}

\item[3.] An unemployed worker did a job search in North Carolina, and found that 15\% of the manufacturing jobs paid 
more than \$17 per hour. What is the standard deviation of the distribution of hourly wage for manufacturing workers in 
North Carolina?

\soln{4cm}{
Given: $\mu = 15.85$, $P(X_{NC} > 17) = 0.15$ \\
This corresponds to a Z score of 1.04 since $P(Z > 1.04) = 0.15$ 
\[ 1.04 = \frac{17 - 15.85}{\sigma} \rightarrow \sigma = \frac{17 - 15.85}{1.04} = 1.11 \]
}

\item[4.] Suppose that a worker applies for a manufacturing job in North Carolina, and receives the good news that she 
got the job and that her pay will be at least \$16.50 per hour. She would really like to be able to make at least \$17 per 
hour. What is the probability that she will get what she wants? Assume that the company she will be working for is a 
run-of-the-mill manufacturing company in NC, i.e. the distribution of the hourly wages at this company reflects the state 
distribution. \textit{Hint:} This is a conditional probability. \\

\soln{6cm}{
Given: $X_{NC} \sim N(\mu = 15.85, \sigma = 1.11)$
\begin{align*}
P(X_{NC} > 17 ~|~ X_{NC} > 16.50) &= \frac{P(X_{NC} > 17 ~and~ X_{NC} > 16.50)}{P(X_{NC} > 16.50)} = \frac{P(X_{NC} > 17)}{P(X_{NC} > 16.50)} \\
&= \frac{0.15}{P\left( Z > \frac{16.50 - 15.85}{1.11} \right)} = \frac{0.15}{P(Z > 0.59)} = \frac{0.15}{0.28} = 0.53
\end{align*}
}

\end{enumerate}

\pagebreak

\textbf{Part 3:}
Government data also indicates that the average hourly wage for manufacturing workers in New York is \$18.39, with a 
standard deviation of \$1.5. 

\begin{enumerate}

\item[5.] Who is doing better within their state: a NC manufacturing worker who makes \$17/hr or a NY manufacturing 
worker who makes \$19/hr?

\soln{4cm}{
\begin{itemize}
\item Given: $X_{NY} \sim N(\mu = 18.39, \sigma = 1.5) \rightarrow X_{NY} = \$19 \rightarrow Z = \frac{19 - 18.39}{1.5} = 0.41$
\item Given: $X_{NC} \sim N(\mu = 15.85, \sigma = 1.11) \rightarrow X_{NC} = \$17 \rightarrow Z = 1.04$
\end{itemize}
NC manufacturing worker doing better since has a higher Z score.
}

\item[6.] If 34\% of NY manufacturing workers make more than \$19/hr, what is the probability that in a random sample of 100 NY manufacturing workers less than 30\% make more than \$19/hr.

\soln{2cm}{
$p = 0.34$, $n = 100$ \\
S/F: checks \\
$\mu = 0.34 \times 100 = 34$ and $\sigma = \sqrt{100 \times 0.34 \times 0.66} = 4.74$ \\
$P \left( K < 30 \right) = P \left( Z < \frac{30 - 34}{4.74} \right) = P(Z < -0.84) = 0.2$
}

\end{enumerate}

\end{document}