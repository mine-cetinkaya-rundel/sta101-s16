\documentclass[12pt]{article}
%%%%%%%%%%%%%%%%
% Packages
%%%%%%%%%%%%%%%%

\usepackage[top=1cm,bottom=0.5cm,left=1.5cm,right= 1.5cm]{geometry}
\usepackage[parfill]{parskip}
\usepackage{graphicx, fontspec, xcolor,multicol, enumitem, setspace}
\DeclareGraphicsRule{.tif}{png}{.png}{`convert #1 `dirname #1`/`basename #1 .tif`.png}

%%%%%%%%%%%%%%%%
% No page number
%%%%%%%%%%%%%%%%

\pagestyle{empty}

%%%%%%%%%%%%%%%%
% User defined colors
%%%%%%%%%%%%%%%%

% Pantone 2015 Fall colors
% http://iwork3.us/2015/02/18/pantone-2015-fall-fashion-report/
% update each semester or year

\xdefinecolor{custom_blue}{rgb}{0, 0.32, 0.48} % FROM SPRING 2016 COLOR PREVIEW
\xdefinecolor{custom_darkBlue}{rgb}{0.20, 0.20, 0.39} % Reflecting Pond  
\xdefinecolor{custom_orange}{rgb}{0.96, 0.57, 0.42} % Cadmium Orange
\xdefinecolor{custom_green}{rgb}{0, 0.47, 0.52} % Biscay Bay
\xdefinecolor{custom_red}{rgb}{0.58, 0.32, 0.32} % Marsala

\xdefinecolor{custom_lightGray}{rgb}{0.78, 0.80, 0.80} % Glacier Gray
\xdefinecolor{custom_darkGray}{rgb}{0.35, 0.39, 0.43} % Stormy Weather

%%%%%%%%%%%%%%%%
% Color text commands
%%%%%%%%%%%%%%%%

%orange
\newcommand{\orange}[1]{\textit{\textcolor{custom_orange}{#1}}}

% yellow
\newcommand{\yellow}[1]{\textit{\textcolor{yellow}{#1}}}

% blue
\newcommand{\blue}[1]{\textit{\textcolor{blue}{#1}}}

% green
\newcommand{\green}[1]{\textit{\textcolor{custom_green}{#1}}}

% red
\newcommand{\red}[1]{\textit{\textcolor{custom_red}{#1}}}

%%%%%%%%%%%%%%%%
% Coloring titles, links, etc.
%%%%%%%%%%%%%%%%

\usepackage{titlesec}
\titleformat{\section}
{\color{custom_blue}\normalfont\Large\bfseries}
{\color{custom_blue}\thesection}{1em}{}
\titleformat{\subsection}
{\color{custom_blue}\normalfont}
{\color{custom_blue}\thesubsection}{1em}{}

\newcommand{\ttl}[1]{ \textsc{{\LARGE \textbf{{\color{custom_blue} #1} } }}}

\newcommand{\tl}[1]{ \textsc{{\large \textbf{{\color{custom_blue} #1} } }}}

\usepackage[colorlinks=false,pdfborder={0 0 0},urlcolor= custom_orange,colorlinks=true,linkcolor= custom_orange, citecolor= custom_orange,backref=true]{hyperref}

%%%%%%%%%%%%%%%%
% Instructions box
%%%%%%%%%%%%%%%%

\newcommand{\inst}[1]{
\colorbox{custom_blue!20!white!50}{\parbox{\textwidth}{
	\vskip10pt
	\leftskip10pt \rightskip10pt
	#1
	\vskip10pt
}}
\vskip10pt
}

%%%%%%%%%%%
% App Ex number    %
%%%%%%%%%%%

% DON'T FORGET TO UPDATE

\newcommand{\appno}[1]
{1.1}

%%%%%%%%%%%%%%
% Turn on/off solutions       %
%%%%%%%%%%%%%%

% Off
\newcommand{\soln}[2]{$\:$\\ \vspace{#1}}{}

%% On
%\newcommand{\soln}[2]{\textit{\textcolor{custom_red}{#2}}}{}

%%%%%%%%%%%%%%%%
% Document
%%%%%%%%%%%%%%%%

\begin{document}
\fontspec[Ligatures=TeX]{Helvetica Neue Light}

Dr. \c{C}etinkaya-Rundel \hfill Sta 101: Data Analysis and Statistical Inference \\
Duke University - Department of Statistical Science \hfill \\

\ttl{Application exercise \appno{}: \\
Distributions of numerical variables}

\inst{Write your responses on a piece of paper. WRITE LEGIBLY! Only one submission per team is required. One team will be randomly selected and their responses will be discussed and graded.}

\tl{Shapes of distributions}

\begin{enumerate}

\item Below are two histograms. One corresponds to the age at which a sample of people applied for marriage licenses; the other corresponds to the last digit of a sample of social security numbers. Which graph is which, and why?

\begin{multicols}{2}

\begin{enumerate}

\item $\:$ \\
\includegraphics[height=1in]{figures/last_digit_SSN}

\item $\:$ \\
\includegraphics[height=1in]{figures/age_mar_lic}

\end{enumerate}

\end{multicols}

\soln{-1cm}{(a) SSN, no pattern expected so uniform (b) age at first marriage, more people get married earlier}

%

\item Match the following variables with the histograms and bar graphs given below. These data represent Sta 101 students at Duke. [\textit{Hint: Think about how each variable should behave.}]

\begin{multicols}{2}
\begin{enumerate}
\item the height of students
\item gender breakdown of students
\item the time it took students to get to their first class of the day
\item the number of hours of sleep students received last night
\item whether or not students live off campus
\item the number of piercings students have
\end{enumerate}
\end{multicols}

\begin{multicols}{3}

\begin{enumerate}

\item[(1)] $\:$ \\
\includegraphics[height=1.2in]{figures/gender}

\item[(2)] $\:$ \\
\includegraphics[height=1.2in]{figures/height}

\item[(3)] $\:$ \\
\includegraphics[height=1.2in]{figures/sleep}

\item[(4)] $\:$ \\
\includegraphics[height=1.2in]{figures/off_campus}

\item[(5)] $\:$ \\
\includegraphics[height=1.2in]{figures/piercings}

\item[(6)] $\:$ \\
\includegraphics[height=1.2in]{figures/time_to_class}

\end{enumerate}

\end{multicols}

\soln{1mm}{(1) gender, (2) height, (3) sleep, (4) off campus, (5) piercings, (6) time to class}

%

\item Come up with a concise way (1-2 sentences) to teach someone how to determine the expected distribution of any variable.

\end{enumerate}

\soln{2cm}{Pay attention to natural boundaries. If skewed, think about whether lower or higher values are more likely.}

%

\tl{Variability}

\begin{enumerate}[resume]

\item[4.] Order histograms A, B, and C from least to most variable. Explain your reasoning.

\includegraphics[width=0.9\textwidth]{figures/histogramsVarOrderABC}

\soln{2cm}{A - least, C - medium, B - most}

\item[5.] Between histograms D and E, which exhibits more variability? Explain your reasoning.

\includegraphics[width=0.9\textwidth]{figures/histogramsVarOrderDE}

\soln{2cm}{E more variable, more observations away from mean}

\end{enumerate}


\end{document}